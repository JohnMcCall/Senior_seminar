% This is a sample document using the University of Minnesota, Morris, Computer Science
% Senior Seminar modification of the ACM sig-alternate style to generate a simple annotated
% bibliography. The idea is that this document is fairly short, consisting of a brief description
% of your sources and how you intend to use them (or not). Most of the ``content'' of the
% generated document comes from the bibliography file, including the notes field which will
% provide the annotations.

% See https://github.com/UMM-CSci/Senior_seminar_templates for more info and to make
% suggestions and corrections.

\documentclass{sig-alternate}
\usepackage{amsmath}

\newtheorem{interactiveProof}{Definition}
\newtheorem{zeroKnowledge}{Definition}

\begin{document}

% --- Author Metadata here ---
%%% REMEMBER TO CHANGE THE SEMESTER AND YEAR
\conferenceinfo{UMM CSci Senior Seminar Conference, December 2013}{Morris, MN}

\title{Zero Knowledge Compilers}

\numberofauthors{1}

\author{
% The command \alignauthor (no curly braces needed) should
% precede each author name, affiliation/snail-mail address and
% e-mail address. Additionally, tag each line of
% affiliation/address with \affaddr, and tag the
% e-mail address with \email.
\alignauthor
John T. McCall\\
	\affaddr{Division of Science and Mathematics}\\
	\affaddr{University of Minnesota, Morris}\\
	\affaddr{Morris, Minnesota, USA 56267}\\
	\email{mcca0798@morris.umn.edu}
}

\maketitle

\begin{abstract}
<Insert Abstract Here>
\end{abstract}

\terms{Need to figure this out yet}

\keywords{Zero Knowledge Protocols, Compilers}

\section{Introduction}
	I will focus on Zero-Knowledge Compilers which are compilers that automatically generate Zero-Knowledge proofs. This is how I plan to use the following sources:
	\begin{itemize}
		\item I expect~\cite{ZKCrypt:2012, Sigma:2009, ZKPDL:2010} to be my core sources, depending on how relevent~\cite{ZKPDL:2010} turns out to be I'll replace it with a better source.
		\item I will use~\cite{MentalGame:1987, Survey, Children:1987} for background information and examples of Zero-Knowledge Protocols. 
		\item I will need to find some papers for background information on compilers.
	\end{itemize}

	As stated above I need to find some sources about compilers. I probably will need to find more papers dealing with ZK-Compilers as well.

	\subsection{Key Points}
	\textbf{What main problems(s) or questions(s) does the research address?}
	
	The main problem that the research address is how to create reliable zero knowledge
	protocols. They can be difficult to define and even harder to verify. Zero knowledge
	compilers help because they can efficiently generate zero knowledge protocols, and
	because of how they are constructed the user can trust that they will work.
	
	\textbf{What are the key contributions of each of your main sources?}
	
	Source~\cite{ZKCrypt:2012} provides a great deal of information about their zero
	knowledge compiler, ZKCrypt. They go into detail about zero knowledge protocols, 
	how their compiler produces them, and they give a proof verifying that their protocols
	are valid. They also talk about a few applications of their compiler.
	
	Source~\cite{ZKPDL:2010} talks in depth about ZKPDL, which is a language they created
	for writing zero knowledge protocols. They also created an interpreter for this
	language, which performs optimizations to lower computational and space overhead.
	This paper also provides an example dealing with electronic cash.
	
	Source~\cite{Sigma:2009} uses $\Sigma$-Protocols in a compiler to automatically 
	generate sound and efficient zero knowledge proofs of knowledge. The compiler
	automatically generates the implementation of the protocol in Java, or it can output
	a description of the protocol in \LaTeX.
	
	\textbf{How are the main sources related to each other?}
	
	The main sources all use compilers to generate zero knowledge protocols, but the
	ways they are implemented are all different so there is some room for comparison.
	All the compilers are also based off $\Sigma$-Protocols, or variations of $\Sigma$-
	Protocols.
	
	\textbf{What is the state of the research?}
	
	The current state is that the compilers have been implemented and tested. They all
	provided enough data to back up their research. Most of the work they are doing now
	will extend the applications of their compilers to support other proof types.
	
	\textbf{What background material will you need to present in order for your audience to understand the research?}
	
	I will need to provide background information on zero knowledge protocols and
	compilers. It's probably more important that I focus on zero knowledge protocols and
	only give basic compiler background.
	


\section{Zero Knowledge Protocols}
	Zero knowledge protocols, also referred to as zero knowledge proofs, are a type
	of protocol in which one party, called the \textit{prover}, tries to convince the 
	other party, called the \textit{verifier}, that a given statement is true. Sometimes
	the statement is that the prover possesses a particular piece of information. This
	is a special case of zero knowledge protocol called a \textit{zero knowledge proof
	of knowledge}~\cite{Wiki}.
	
	\subsection{Definition}
		Formally, a zero knowledge proof is a type of interactive proof. Mohr gives
		a concise definition in~\cite{Survey}.
	
		\begin{interactiveProof}
			An interactive proof system for a set $S$ is a two-party game between a
			\textit{verifier} executing a probabilistic polynomial-time strategy and
			a \textit{prover} which executes a computationally unbound strategy 
			satisfying:
			
			\begin{itemize}
				\item \textit{Completeness}: For every $ x \in S$, the verifier always
				accepts after interacting with the prover on common input $x$
				
				\item \textit{Soundness}: For some polynomial $p$, it holds that for
				every $x \notin S$ and every potential strategy $P^{*}$, the verifier
				rejects with probability at least $\frac{1}{p(|x|)}$ after interacting
				with $P^{*}$ on common input $x$.
			\end{itemize}
			
		\end{interactiveProof}
		
		To summarize: if an honest verifier is always convinced after interacting with
		an honest prover, then the proof is complete. A proof is sound if a cheating
		prover can only convince an honest verifier with some small probability.
		
		For an interactive proof to be a zero knowledge proof it must
		satisfy the condition of \textit{zero knowledge}.		
		
		\begin{zeroKnowledge}
			A proof is zero knowledge on (inputs from) the set $S$ if, for every feasible
			strategy $B^{*}$ there exists a feasible computation $C^{*}$ so that the
			following two probability ensembles are computationally indistinguishable:
			
			\begin{itemize}
				\item the output of $B^{*}$ after interacting with $A$ on common input
				$x \in S$
				
				\item the output of $C^{*}$ on input $x \in S$
			\end{itemize}						
			
		\end{zeroKnowledge}
		
		In other words, any information, in this case $B^{*}$, that can be learned by
		interacting with $A$ can also be learned without interacting with
		$A$.~\cite{Survey}

	\subsection{Examples}
	%Write an example using the Lyrics of Billie Jean	
	
	Below are two examples of zero knowledge protocols. The first
	example is easy to follow and just highlights how a zero knowledge
	protocol functions. The second example is more in depth and shows
	how an application of zero knowledge protocols.
	
	\subsubsection{The Magic Cave}
	The classic example for zero knowledge protocols is the cave example.
	First presented in~\cite{Children:1987} and then restated
	in~\cite{Survey} the cave example is the go-to example for learning
	zero knowledge protocols.

	Peggy has stumbled across a magical cave. Upon entering the cave
	there are two paths, one leading to the right and one leading to the
	left. Both paths eventually lead to a dead end, however Peggy has
	discovered a secret word that opens up a hidden door in the dead end,
	connecting both paths.

	Victor hears about this, and offers to buy the secret from Peggy for
	\$1,000,000, which Peggy agrees to. Before giving Peggy the money Victor
	wants to be certain that Peggy actually knows this secret word. How can
	Peggy (the prover) convince Victor (the verifier) that she knows the
	word, without revealing what it is?

	The two of them come up with the following plan. First, Victor will wait
	outside the cave while Peggy goes in. She will randomly pick either the
	right or the left path and go down it. Since Victor was outside he
	should have no knowledge of which path Peggy took. Then Victor will
	enter the cave. He will wait by the fork and shout to Peggy which
	path to return from. 
	
	Assuming that Peggy knows the word, she should be able to return down
	the correct path, regardless of which one she started on. If Victor 
	says to	return down the path she started on, she simply walks back. 
	If Victor says to return down the other path, she whispers the magic
	word, goes through the door, and returns down the other path.

	If Peggy doesn't know the word, about there is a 50\% chance that Victor
	will choose the path she did not start down. If this happens there is
	no way that she can return down the correct path. The experiment should
	be repeated until Victor either discovers Peggy is a liar because she
	returned down the wrong path, or until he is sufficiently satisfied
	that she does indeed know the word.

	This is a zero knowledge protocol because it satisfies each of the three
	requirements. It satisfies completeness because	if Peggy knows the word
	she will be able to convince Victor. It is sound because if Peggy does not 
	know the word, she will not be able to convince Victor unless she was very
	lucky. Finally it is zero knowledge because if Victor follows the protocol
	he will not be able to learn anything besides whether or not Peggy knows 
	the word.

	\subsubsection{Zero-Knowledge Proof for Graph Isomorphism}

\section{Compilers}
	There are many different types of compilers, single-pass compilers, multi-pass,
	load-and-go, debugging compilers, optimizing compilers, and many combinations
	of these. Different compilers do different things, but at their core all 
	compilers must perform one function. Simply put, they must take a program as 
	an input and output	an equivalent program in a different
	language.~\cite{Compiler:1986} 
	
	The first compilers started to appear in the 1950s. Much of the early work
	dealt with translating arithmetic formulas into machine code. At the time
	compilers were notoriously difficult to implement, for instance it took
	18 staff-years to implement the first Fortran compiler. Various languages,
	programming	environments, and tools have been developed since then which
	make implementing a compiler considerably easier.
	
	There are two parts to compilation, analysis and synthesis. Analysis breaks
	up the source into pieces and creates an intermediate representation, usually
	a syntax tree, of the program. Synthesis constructs the target program from
    the representation. There are a few different types of analysis that a compiler
    can perform, such as:
    
    \begin{enumerate}
    \item \textit{Linear (or Lexical) analysis}, in which the stream of characters
    making up the source program is read from left-to-right and grouped into 
    \textit{tokens} that are sequences of characters having a collective meaning.
    
    \item \textit{Hierarchical analysis}, in which characters or tokens are grouped
    hierarchically into nested collections with collective meaning.
    
    \item \textit{Semantic analysis}, in which certain checks are performed to ensure
    that the components of a program fit together meaningfully.
    \end{enumerate}
    
    

\section{Zero Knowledge Compilers}
	This section will be the main section. Here I will talk about my core sources and
	how they are using their compilers.

	\subsection{Sigma-Protocols}
	% I want to touch this area up and make it more clear just what Sigma
	% Protocols are. I also want to talk about the compiler that the authors made
		$\Sigma$-Protocols are the basis of essentially all efficient zero knowledge
		proofs of knowledge used in practice today. Using a homomorphism, such as an
		RSA function, one can use $\Sigma$-protocols to prove knowledge of a secret
		preimage. There are several variations of these proofs. One such variation is
		the "AND-proof" which allows simultaneous proving of multiple preimages under
		different homomorphisms. There are also "OR-proofs" and proofs which show
		that different preimages fulfill a set of linear relations.~\cite{Sigma:2009}
		
		Many practical applications use techniques based on $\Sigma$-protocols. Examples
		include identification schemes, interactive verifiable computation, group
		signatures, secure watermark detection, and efficient secure multiparty
		computation. Many of these applications exist only at the specification level,
		however real-world applications using zero knowledge proofs of knowledge have
		recently been produced.
		
	\subsection{ZKPDL}
		Meiklejohn et al. provide a language called the Zero-Knowledge Proof Description
		Language (ZKPDL)~\cite{ZKPDL:2010}. This language makes it much easier for both
		programmers and 	cryptographers to implement protocols. The authors also provide
		a library called Cashlib, which provides the language access to other simple
		cryptographic protocols such as electronic cash, blind signatures, verifiable
		encryption and fair exchange.
		
		This language provides two main benefits. Firstly, the programmer no longer 
		has to worry about implementing cryptographic primitives, efficient mathematical
		operations, or generating and processing messages. ZKPDL allows the user to
		specify the protocol similarly to how it would be specified in a theoretical
		description. Secondly, the library makes performance optimizations based on an
		analysis of the protocol description. 
		
		The authors also provide an interpreter for ZKPDL, implemented in C++, which
		preforms one of two actions depending on the role of the user. On the prover
		side it outputs a zero knowledge proof. On the verifier side it takes a proof
		and verifies its correctness. Regardless of the role of the user, the program
		given to the interpreter is the same. The interpreter also performs a number of
		optimizations including precomputations, caching, and translations to prevent
		redundant proofs. 
		
		A program written in this language is split into two blocks: a computation block,
		and a proof block. Both blocks are optional, if the user is only interested in the
		computation they can just write that. Alternatively, if the user has all the computations
		done they can just write the proof block. Here is a sample of code written in ZKPDL.
		
		\begin{verbatim}
		computation: // compute values required for proof
		  given: // declarations
		    group: G = <g. h>
		    exponents in G: x[2:3]
		  compute: // declarations and assignments
		    random exponents in G: r[1:3]
		    x_1 := x_2 * x_3
		    for(i, 1:3, c_i := g^x_i * h^r_i)
		    
		proof:
		  given: // declarations of public values
		    group: G = <g, h>
		    elements in G: c[1:3]
		    for(i, 1:3, commitment to x_i: c_i = g^x_i * h^r_i)
		  prove knowledge of: // declarations of private values
		    exponents in G: x[1:3], r[1:3]
		  such that: // protocol specification; i.e. relations
		    x_1 = x_2 * x_3
		\end{verbatim}
		
		In this example, the authors are proving that the value $x_{1}$ contained within
		the commitment $c_{1}$ is the product of $x_{2}$ and $x_{3}$ which are contained
		in $c_{2}$ and $c_{3}$ respectively. Because both blocks are optional, they are
		considered independent from each other, so a lines are repeated between the two.
		
	\subsection{ZKCrypt}
		Here I'll talk about ZKCrypt, it's implementation and verification steps.

\section{Applications}
	In general, zero knowledge protocols have many applications. Authentication systems,
	electronic voting, electronic ticketing, Direct Anonymous Attestation (DAA), and
	Off-the-Record messaging~\cite{ZKCrypt:2012, ZKPDL:2010} are just a few examples.
	The applications that will be focused on in this paper are electronic cash, and
	deniable authentication.
	
	\subsection{Electronic Cash}
	Electronic Cash, or e-cash, is an electronic currency. E-cash maintains the buyer's
	anonymity, unlike a debit or credit card that is used to purchase something 
	electronically. Bitcoins are a recent example of an e-cash system.
	
	Okamota and Ohta describe the ideal electronic cash system in~\cite{Ecash:1991}.
	The ideals are as follows:
	
	\begin{enumerate}
		\item \textit{Independence}: The security of electronic cash cannot depend on any
		physical condition. Then the cash can be transferred through networks.
		
		\item \textit{Security}: The ability to copy (reuse) and forge the cash must be
		prevented.
		
		\item \textit{Privacy (Untraceability)}: The privacy of the user should be
		protected. That is, the relationship between the user and their purchases must
		be untraceable by anyone.
		
		\item \textit{Off-line payment}: When a user pays the electronic cash to a shop, 
		the procedure between the user and the shop should be executed in an off-line
		manner. That is, the shop does not need to be linked to the host in user's
		payment procedure.
		
		\item \textit{Transferability}: The cash can be transferred to other users.
		
		\item \textit{Dividability}: One issued piece of cash worth value $C$ (dollars)
		can be subdivided into many pieces such that each subdivided piece is worth any
		desired value less than $C$ and the total value of all the pieces is equivalent
		to $C$.
	\end{enumerate}
	
	Almeida et al. describe briefly in~\cite{ZKCrypt:2012} how ZKCrypt can be used to
	generate a proof for proving the identity of the user when withdrawing money from
	a bank account. The user has to prove they have a secret key in order to successfully
	withdraw money. 
	
	The authors state the proof goal as:
	\begin{align*}
	ZPK\left[(u_{1}, u_{2}): I = g^{u_{1}}_{1}g^{u_{2}}_{2}\right].
	\end{align*}
	In this goal, $I, g_{1}, g_{2} \in \mathbb{Z}^{*}_{p}$ such that ord $g_{1}$ = ord
	$g_{2}$ = $q$, where $q|(p - 1)$ and $p, q \in \mathbb{P}$. The secrets $u_{1}, u_{2}$
	are elements of $\mathbb{Z}_{q}$. A single instance of the $\Sigma^{\Phi}$-protocol
	is enough to realize this goal.
	
	Meiklejohn et al. also give this example, a user proving their identity to the bank,
	implemented in ZKDPL. The program for this looks like: 
	\begin{verbatim}
	proof:
	  given:
	  	group: cashGroup = <f, g, h, h1, h2>
	  	elements in cashGroup: A, pk_u
	  	  commitment to sk_u: A = g^sk_u * h^r_u
	  prove knowledge of:
	  	exponents in cashGroup: sk_u, r_u
	  such that:
	  	pk_u = g^sk_u
	  	A = g^sk_u * h^r_u
	\end{verbatim}
	
	When the bank has verified this proof, the bank and the user will run a protocol
	which defines a wallet which contains $W$ coins, where $W$ is a system-wide
	public parameter. When a user spends a coin, it is split up into two parts: an
	endorsed part and an unendorsed part. Separately the two parts are worthless, but
	together the coin becomes valid. First the unendorsed part is sent to the vendor
	who proves its validity. The vendor then sends what the buyer has purchased. The
	buyer sends the endorsed portion of the coin to the vendor upon receiving their
	product.
	
	\subsection{Deniable Authentication}

\section{Conclusion}
	This is where I'll neatly wrap everything up.

% The following two commands are all you need to
% produce the bibliography for the citations in your paper.
\bibliographystyle{abbrv}
% annotated_bibliography.bib is the name of the BibTex file containing 
% all the bibliography entries for this example. Note that you *don't* include the .bib ending
% in the \bibliography command.
\bibliography{annotated_bibliography}

% You must have a ".bib" file and remember to run:
%     pdflatex bibtex pdflatex pdflatex
% in order to see all the citation references correctly.

\end{document}