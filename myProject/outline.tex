% This is a sample document using the University of Minnesota, Morris, Computer Science
% Senior Seminar modification of the ACM sig-alternate style to generate a simple annotated
% bibliography. The idea is that this document is fairly short, consisting of a brief description
% of your sources and how you intend to use them (or not). Most of the ``content'' of the
% generated document comes from the bibliography file, including the notes field which will
% provide the annotations.

% See https://github.com/UMM-CSci/Senior_seminar_templates for more info and to make
% suggestions and corrections.

\documentclass{sig-alternate}

\begin{document}

% --- Author Metadata here ---
%%% REMEMBER TO CHANGE THE SEMESTER AND YEAR
\conferenceinfo{UMM CSci Senior Seminar Conference, December 2013}{Morris, MN}

\title{Zero Knowledge Compilers}

\numberofauthors{1}

\author{
% The command \alignauthor (no curly braces needed) should
% precede each author name, affiliation/snail-mail address and
% e-mail address. Additionally, tag each line of
% affiliation/address with \affaddr, and tag the
% e-mail address with \email.
\alignauthor
John T. McCall\\
	\affaddr{Division of Science and Mathematics}\\
	\affaddr{University of Minnesota, Morris}\\
	\affaddr{Morris, Minnesota, USA 56267}\\
	\email{mcca0798@morris.umn.edu}
}

\maketitle

\begin{abstract}
<Insert Abstract Here>
\end{abstract}

\terms{Need to figure this out yet}

\keywords{Zero Knowledge Protocols, Compilers}

\section{Introduction}
	I will focus on Zero-Knowledge Compilers which are compilers that automatically generate Zero-Knowledge proofs. This is how I plan to use the following sources:
	\begin{itemize}
		\item I expect~\cite{ZKCrypt:2012, Sigma:2009, ZKPDL:2010} to be my core sources, depending on how relevent~\cite{ZKPDL:2010} turns out to be I'll replace it with a better source.
		\item I will use~\cite{MentalGame:1987, Survey, Children:1987} for background information and examples of Zero-Knowledge Protocols. 
		\item I will need to find some papers for background information on compilers.
	\end{itemize}

	As stated above I need to find some sources about compilers. I probably will need to find more papers dealing with ZK-Compilers as well.

	\subsection{Key Points}
	\textbf{What main problems(s) or questions(s) does the research address?}
	
	The main problem that the research address is how to create reliable zero knowledge
	protocols. They can be difficult to define and even harder to verify. Zero knowledge
	compilers help because they can efficiently generate zero knowledge protocols, and
	because of how they are constructed the user can trust that they will work.
	
	\textbf{What are the key contributions of each of your main sources?}
	
	Source~\cite{ZKCrypt:2012} provides a great deal of information about their zero
	knowledge compiler, ZKCrypt. They go into detail about zero knowledge protocols, 
	how their compiler produces them, and they give a proof verifying that their protocols
	are valid. They also talk about a few applications of their compiler.
	
	Source~\cite{ZKPDL:2010} talks in depth about ZKPDL, which is a language they created
	for writing zero knowledge protocols. They also created an interpreter for this
	language, which performs optimizations to lower computational and space overhead.
	This paper also provides an example dealing with electronic cash.
	
	Source~\cite{Sigma:2009} uses $\Sigma$-Protocols in a compiler to automatically 
	generate sound and efficient zero knowledge proofs of knowledge. The compiler
	automatically generates the implementation of the protocol in Java, or it can output
	a description of the protocol in \LaTeX.
	
	\textbf{How are the main sources related to each other?}
	
	The main sources all use compilers to generate zero knowledge protocols, but the
	ways they are implemented are all different so there is some room for comparison.
	All the compilers are also based off $\Sigma$-Protocols, or variations of $\Sigma$-
	Protocols.
	
	\textbf{What is the state of the research?}
	
	The current state is that the compilers have been implemented and tested. They all
	provided enough data to back up their research. Most of the work they are doing now
	will extend the applications of their compilers to support other proof types.
	
	\textbf{What background material will you need to present in order for your audience to understand the research?}
	
	I will need to provide background information on zero knowledge protocols and
	compilers. It's probably more important that I focus on zero knowledge protocols and
	only give basic compiler background.
	


\section{Zero Knowledge Protocols}
	The section will cover Zero Knowledge Protocols and will provide background
	and examples. There will probably be an easy to grasp example, such as the cave
	example, and a more advanced example.
	
	\subsection{Background}

	\subsection{Examples}


\section{Compilers}
	This section will provide some basic background information about compilers.

	\subsection{Background}
	

\section{Zero Knowledge Compilers}
	This section will be the main section. Here I will talk about my core sources and
	how they are using their compilers.

	\subsection{Sigma-Protocols}
		Here I'll talk about Sigma-protocols and how they are used in the following
		compilers.
		
		Note to myself: Look into the Fiat-Shamir heuristic
	\subsection{ZKCrypt}
		Here I'll talk about ZKCrypt, it's implementation and verification steps.
		
	\subsection{ZKPDL}
		Similar to ZKCrypt section.

\section{Applications}
	Here I'll talk about how those compilers are used in the real world.
	
	\subsection{Electronic Cash}
	
	\subsection{Another Application}

\section{Conclusion}
	This is where I'll neatly wrap everything up.

% The following two commands are all you need to
% produce the bibliography for the citations in your paper.
\bibliographystyle{abbrv}
% annotated_bibliography.bib is the name of the BibTex file containing 
% all the bibliography entries for this example. Note that you *don't* include the .bib ending
% in the \bibliography command.
\bibliography{annotated_bibliography}

% You must have a ".bib" file and remember to run:
%     pdflatex bibtex pdflatex pdflatex
% in order to see all the citation references correctly.

\end{document}



